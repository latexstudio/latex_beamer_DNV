
%%%%%%%%%%%%%%%%%%%%%%%%%%%%%%%%%%%%%%%%%%%%%%%%%%%
%%%%	Blocks, Alerts \& Math Environments
%%%%%%%%%%%%%%%%%%%%%%%%%%%%%%%%%%%%%%%%%%%%%%%%%%%
\section{Blocks}
\begin{frame}{Blocks, Alerts \& Math Environments}
	\begin{block}{Notation}
		This is some notation.
	\end{block}

	\begin{definition}
		This is a definition.
	\end{definition}
	
	\begin{remark}
		And this is a remark.
	\end{remark}
\end{frame}

\begin{frame}
	\begin{theorem}[Existence \& Uniqueness for ...]
		A theorem is important so it should be emphasised!
	\end{theorem}

	\begin{proposition}
		A proposition may be a little less important but it's also worth emphasising!
	\end{proposition}
	
	In the same way we can create lemmata \& corollaries.
\end{frame}

\begin{frame}\label{slide::ex-alert-el}
	\begin{example}
		Examples should be made. Of course normally only the mathematician understands why this text on the blackboard should be an example.
	\end{example}

	Finally
		
	\begin{alertblock}{Alert, alert}
		An alert block has a catchy colour.
	\end{alertblock}
\end{frame}


{
%	\definecolor{dnvdarkblue}{RGB}{137,57,94}
%	\definecolor{dnvgreen}{RGB}{57,137,100}
%	\definecolor{dnvskyblue}{RGB}{255,175,0}
	\begin{frame}{Red purple colour scheme}
		Finally let us have a look at the second colour scheme:
		
		\begin{theorem}[Existence \& Uniqueness for ...]
			A theorem is important so it should be emphasised!
		\end{theorem}
	
		\begin{example}
			Examples should be made. Of course normally only the mathematician understands why this text on the blackboard should be an example.
		\end{example}
	
		\begin{alertblock}{Alert, alert}
			An alert block has a catchy colour.
		\end{alertblock}
	\end{frame}
}


%%%%%%%%%%%%%%%%%%%%%%%%%%%%%%%%%%%%%%%%%%%%%%%%%%%
%%%%	Overlays \& Images
%%%%%%%%%%%%%%%%%%%%%%%%%%%%%%%%%%%%%%%%%%%%%%%%%%%
\section{Overlays \& Images}
\begin{frame}[fragile]
	\frametitle{Overlays \& Images}
	Complying with the old saying: \enquote{A picture speaks a thousand words}, we can create frames which only contain a picture by
	\begin{center}
		\texttt{\textbackslash imageFrame\{imageURL\}.}
	\end{center}
	We can also define an overlay on the left oder right side of the frame. By default the overlay has a width of 150pt. We can adjust as an optional argument.
	\begin{verbatim} \imageFrameOverlayLeft[optional width]{%
    ./gfx/horizontallift.pdf}{%
    Want big impact?}{%
    Use a big picture.} \end{verbatim}
\end{frame}

%%%%	image frame with overlay left
\imageFrameOverlayLeft{%
./gfx/horizontallift.pdf}{%
Want big impact?}{%
Use a big picture.}

%%%%	image frame with overlay right, custom size
\imageFrameOverlayRight[150pt]{%
./gfx/horizontallift.pdf}{%
Need a bigger overlay on the right side?}{%
}


%%%%%%%%%%%%%%%%%%%%%%%%%%%%%%%%%%%%%%%%%%%%%%%%%%%
%%%%	Listings, Tables, Highlighted Text \& Tikz
%%%%%%%%%%%%%%%%%%%%%%%%%%%%%%%%%%%%%%%%%%%%%%%%%%%
\section{Listings, Tables, Highlighted Text \& Tikz}

\begin{frame}{Listings}
	\begin{itemize}
		\item Item $\sharp$1
		\item Item $\sharp$2
			\begin{itemize}
				\item Subitem 2.$\sharp$1
				\item Subitem 2.$\sharp$2
			\end{itemize} 
	\end{itemize}
	and so on. We can also create highlighted lists:
	\begin{itemize}[<+- | alert@+>]
		\item \alert<4>{\only<-3>{Hi!}\only<4>{or here?}}
		\item you
		\item there!
	\end{itemize}
\end{frame}

\begin{frame}{Tables}
	\begin{center}
		\arrayrulecolor{dnvdarkblue}
		\begin{tabular}[]{lrrl}
			\toprule
								& \multicolumn{1}{c}{{\bfseries Dual space}}
			                    & \multicolumn{1}{c}{{\bfseries Reflexive}}
			                    & \multicolumn{1}{l}{{\bfseries Norm}} \\
			\midrule
			$\mathbb K^n$	& $\mathbb K^n$			& Yes			& $\Vert x\Vert_2 =\left(\sum_{i=1}^n |x_i|^2\right)^{\frac 12}$\\[0.5em]
			$\ell_p$			& $\ell_q$				& Yes			& $\Vert x\Vert_p = \left( \sum_{i=1}^\infty |x_i|^p \right)^{\frac 1p}$\\[0.5em]
			$\ell_1$			& $\ell_\infty$			& \alert{No}	& $\Vert x\Vert_1 = \sum_{i=1}^\infty |x_i|$\\[0.5em]
			$\ell_\infty$	& {\small complicated}	& \alert{No}	& $\Vert x\Vert_\infty = \sup_i |x_i|$\\[0.5em]
			$L^p(\mu)$		& $L^q(\mu)$			& Yes			& $\Vert f\Vert_p = \left( \int |f|^p \, \mathrm d\mu \right)^{\frac 1p}$\\[0.5em]
			$L^1(\mu)$		& $L^\infty(\mu)$ 		& \alert{No} 	& $\Vert f\Vert_1 = \int |f| \, \mathrm d\mu$\\
			\bottomrule
		\end{tabular}
	\end{center}
\end{frame}

%%%%%%%%%%%%%%%%%%%%%%%%%%%%%%%%%%%%%%%%%%%%%%%%%%%
%%%%	highlighted frame with number and 
%%%%	optional subtext
%%%%%%%%%%%%%%%%%%%%%%%%%%%%%%%%%%%%%%%%%%%%%%%%%%%
\highlightedFrame[such an impressive number should be big!]{89.432.567}

\begin{frame}{Tikz}
	\begin{center}
	    \begin{tikzpicture}[scale=.8]
	        \pgfmathsetseed{2236}
	
	        \fill (0,0) circle (2pt);
		    \draw (0,0) ellipse (5 and 3);
	    		\node at (5,-2.5) {$D \subsetneq \mathbb R^n$};
	        
	        \draw[decorate, decoration={random steps,segment length=5pt,amplitude=10pt}] [dnvdarkblue] plot [smooth, tension=1] coordinates { (0,0) (2,0) (0,2.5) (-3,0) (0,-1.5) (4,1.8)};
	        \node[dnvdarkblue] at (2.8,-1.5) {BM with $p_t$, $T_t$};
	        
	        \node at (5.2,1.8) {$\partial D \ni x_0$};
	        \fill (4,1.8) circle (1pt);
	        
	        \draw[->,dnvgreen,thick] (4,1.8) -- (3,1);
	        
	        \draw[->,dnvskyblue,thick] (4,1.8) -- (5,2.6);
	        \fill (4,1.8) circle (.5pt);
	    \end{tikzpicture}
	\end{center}
	
	\begin{itemize}
	    \item Trap, $x_0$ absorbing = Killing Brownian motion = Dirichlet problem
	    \item {\color{dnvgreen} Reflected Brownian motion = Neumann problem}
	    \item {\color{dnvskyblue} Wait an go on = Sticky Brownian motion}
	\end{itemize}
\end{frame}
